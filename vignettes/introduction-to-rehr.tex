\documentclass[]{article}
\usepackage{lmodern}
\usepackage{amssymb,amsmath}
\usepackage{ifxetex,ifluatex}
\usepackage{fixltx2e} % provides \textsubscript
\ifnum 0\ifxetex 1\fi\ifluatex 1\fi=0 % if pdftex
  \usepackage[T1]{fontenc}
  \usepackage[utf8]{inputenc}
\else % if luatex or xelatex
  \ifxetex
    \usepackage{mathspec}
    \usepackage{xltxtra,xunicode}
  \else
    \usepackage{fontspec}
  \fi
  \defaultfontfeatures{Mapping=tex-text,Scale=MatchLowercase}
  \newcommand{\euro}{€}
\fi
% use upquote if available, for straight quotes in verbatim environments
\IfFileExists{upquote.sty}{\usepackage{upquote}}{}
% use microtype if available
\IfFileExists{microtype.sty}{%
\usepackage{microtype}
\UseMicrotypeSet[protrusion]{basicmath} % disable protrusion for tt fonts
}{}
\usepackage[margin=1in]{geometry}
\usepackage{color}
\usepackage{fancyvrb}
\newcommand{\VerbBar}{|}
\newcommand{\VERB}{\Verb[commandchars=\\\{\}]}
\DefineVerbatimEnvironment{Highlighting}{Verbatim}{commandchars=\\\{\}}
% Add ',fontsize=\small' for more characters per line
\usepackage{framed}
\definecolor{shadecolor}{RGB}{248,248,248}
\newenvironment{Shaded}{\begin{snugshade}}{\end{snugshade}}
\newcommand{\KeywordTok}[1]{\textcolor[rgb]{0.13,0.29,0.53}{\textbf{{#1}}}}
\newcommand{\DataTypeTok}[1]{\textcolor[rgb]{0.13,0.29,0.53}{{#1}}}
\newcommand{\DecValTok}[1]{\textcolor[rgb]{0.00,0.00,0.81}{{#1}}}
\newcommand{\BaseNTok}[1]{\textcolor[rgb]{0.00,0.00,0.81}{{#1}}}
\newcommand{\FloatTok}[1]{\textcolor[rgb]{0.00,0.00,0.81}{{#1}}}
\newcommand{\CharTok}[1]{\textcolor[rgb]{0.31,0.60,0.02}{{#1}}}
\newcommand{\StringTok}[1]{\textcolor[rgb]{0.31,0.60,0.02}{{#1}}}
\newcommand{\CommentTok}[1]{\textcolor[rgb]{0.56,0.35,0.01}{\textit{{#1}}}}
\newcommand{\OtherTok}[1]{\textcolor[rgb]{0.56,0.35,0.01}{{#1}}}
\newcommand{\AlertTok}[1]{\textcolor[rgb]{0.94,0.16,0.16}{{#1}}}
\newcommand{\FunctionTok}[1]{\textcolor[rgb]{0.00,0.00,0.00}{{#1}}}
\newcommand{\RegionMarkerTok}[1]{{#1}}
\newcommand{\ErrorTok}[1]{\textbf{{#1}}}
\newcommand{\NormalTok}[1]{{#1}}
\usepackage{longtable,booktabs}
\ifxetex
  \usepackage[setpagesize=false, % page size defined by xetex
              unicode=false, % unicode breaks when used with xetex
              xetex]{hyperref}
\else
  \usepackage[unicode=true]{hyperref}
\fi
\hypersetup{breaklinks=true,
            bookmarks=true,
            pdfauthor={David A Springate\^{}\{*1\}, Rosa Parisi\^{}\{2\}, \ldots{}, Evangelos Kontopantelis\^{}\{1\}; \^{}\{1\}Institute of Population Health, University of Manchester; \^{}\{2\}Centre for Pharmacoepidemiology and Drug Safety Research, Manchester Pharmacy School, University of Manchester; \^{}\{*\}Corresponding Author},
            pdftitle={rEHR: An R package for manipulating and analysing Electronic Health Record data},
            colorlinks=true,
            citecolor=blue,
            urlcolor=blue,
            linkcolor=magenta,
            pdfborder={0 0 0}}
\urlstyle{same}  % don't use monospace font for urls
\setlength{\parindent}{0pt}
\setlength{\parskip}{6pt plus 2pt minus 1pt}
\setlength{\emergencystretch}{3em}  % prevent overfull lines
\setcounter{secnumdepth}{0}

%%% Use protect on footnotes to avoid problems with footnotes in titles
\let\rmarkdownfootnote\footnote%
\def\footnote{\protect\rmarkdownfootnote}

%%% Change title format to be more compact
\usepackage{titling}
\setlength{\droptitle}{-2em}
  \title{rEHR: An R package for manipulating and analysing Electronic Health
Record data}
  \pretitle{\vspace{\droptitle}\centering\huge}
  \posttitle{\par}
  \author{David A Springate\(^{*1}\), Rosa Parisi\(^{2}\), \ldots{}, Evangelos
Kontopantelis\(^{1}\) \\ \(^{1}\)Institute of Population Health, University of Manchester \\ \(^{2}\)Centre for Pharmacoepidemiology and Drug Safety Research,
Manchester Pharmacy School, University of Manchester \\ \(^{*}\)Corresponding Author}
  \preauthor{\centering\large\emph}
  \postauthor{\par}
  \predate{\centering\large\emph}
  \postdate{\par}
  \date{2016-01-05}




\begin{document}

\maketitle

\begin{abstract}
Research with structured Electronic Health Records (EHRs) is expanding
as data becomes more accessible; analytic methods advance; and the
scientific validity of such studies is increasingly accepted. However,
data science methodology to enable the rapid searching/extraction,
cleaning and analysis of these large, often complex, datasets is less
well developed. In addition, commonly used software is inadequate,
resulting in bottlenecks in research workflows and in obstacles to
increased transparency and reproducibility of the research. Preparing a
research-ready dataset from EHRs is a complex and time consuming task
requiring substantial data science skills, even for simple designs. In
addition, certain aspects of the workflow are computationally intensive,
for example extraction of longitudinal data and matching controls to a
large cohort, which may take days or even weeks to run using standard
software. The rEHR package simplifies and accelerates the process of
extracting ready-for-analysis datasets from EHR databases. It has a
simple import function to a database backend that greatly accelerates
data access times. A set of generic query functions allow users to
extract data efficiently without needing detailed knowledge of SQL
queries. Longitudinal data extractions can also be made in a single
command, making use of parallel processing. The package also contains
functions for cutting data by time-varying covariates, matching controls
to cases, unit conversion and construction of clinical code lists. There
are also functions to synthesise dummy EHR. The package has been tested
with one for the largest primary care EHRs, the Clinical Practice
Research Datalink (CPRD), but allows for a common interface to other
EHRs. This simplified and accelerated work flow for EHR data extraction
results in simpler, cleaner scripts that are more easily debugged,
shared and reproduced.
\end{abstract}

\section{1. Introduction}\label{introduction}

We present the \texttt{R} (R Core Team 2014) package \texttt{rEHR} for
manipulating and analysing Electronic Health Record (EHR) data and
demonstrate its use with rEHR-generated synthetic data. \texttt{rEHR} is
available from the Comprehensive R Archive Network (CRAN) at
\textbf{{[}{[}upload to CRAN!{]}{]}}.

The package has been developed using structured primary care data from
the UK, which has enjoyed near-universal deployment of EHRs in general
practice and clinical coding performed by general practitioners for over
twenty years. Comprehensive extracts of these UK primary care records
are made available for research - the main sources are: The Clinical
Practice Research Datalink (CPRD, previously known as the General
Practice Research Database, GPRD), The Health Improvement Network
(THIN), QResearch, The Doctors' Independent Network (DIN-LINK) and more
recently, Research One. These databases hold near complete medical
records for millions of patients. To date, over 1600 papers have
published using these UK primary care databases (PCDs), with well over
150 papers published per year since 2012. EHR research is set to grow
still faster due to advances in analysis methodology (e.g.
({\textbf{???}})), an increasing body of evidence supporting the
validity of such studies (e.g. Reeves et al. (2014), Springate et al.
(2015)) and efforts to improve transparency and reproducibility (D. A.
Springate et al. (2014)).

Despite the research interest in PCDs, data science methodology to
enable the rapid searching/extraction, cleaning and analysis of these
increasingly large and complex datasets is less well developed. In
addition, commonly used software tools are often inadequate, resulting
in bottlenecks in the research workflow and in obstacles to increased
transparency and reproducibility of research. PCDs such as CPRD store
data in complex relational and nested structures, and preparing an
analysis-ready dataset requires substantial data science skills, even
for simple designs. This complexity is an inevitable consequence of the
wide range of information contained within these databases, which detail
the primary care history for every patient, including coded data for all
diagnoses, prescriptions, referrals and test results for all
consultations. To manage this vast wealth of data requires a relational
structure based on multiple tables, classifications and terminologies
(e.g.~Read codes for diagnoses and referrals, product codes for
prescriptions). To extract relevant data, research teams have to
complete a sequence of non-trivial technical tasks. The more complex the
research design the more steps are required to obtain the final dataset.
For example, investigating drug outcomes typically involves constructing
complex definitions of codes for diagnosis, drug exposure (may be
varying over time), mortality, and possible confounding factors
(e.g.~comorbidities, additional medications, gender, age, referrals,
date of diagnosis, etc.). In addition, certain aspects of the workflow
are computationally intensive (for example extraction of longitudinal
data and matching controls to a large cohort) - often taking days or
even weeks to run using standard software. Although more powerful
compute facilities help (and are practically a prerequisite for working
with these data), an inefficient and slow program running on a fast
server will still be inefficient and slow. Some `how-to' papers exist
for good practice in observational data management but they address only
some of the issues or focus on specific applications (Danaei et al.
2013; Davé and Petersen 2009; J. M. Overhage and Overhage 2013; Perlis
et al. 2012). At the same time there is a wealth of health informatics
and computer science literature on how to make these research processes
more transparent, reducing the duplication of effort and improving the
consistency of data processing (Ainsworth, Cunningham, and Buchan 2012;
Bechhofer et al. 2013). Finally, several software packages exist for
speeding up data analysis, but these are generic, do not apply directly
to EHR manipulation and may require specialist knowledge to effectively
use (e.g. \texttt{dplyr} (Wickham and Francois 2015) for fast
manipulation of dataframes \texttt{sqldf} (Grothendieck 2014) for
database integration and \texttt{parallel} (in base \texttt{R}) for
parallel processing).

\texttt{rEHR} simplifies and accelerates the process of extracting
ready-for-analysis datasets from EHR databases. In section 2 we provide
instructions on loading the software and importing flat text files of
the kind supplied by EHR providers into a local SQL database. In section
3 we describe the basic query operations provided by the package, the
building of longitudinal data and calculation of prevalence and
incidence statistics. In section 4 we convert the longitudinal data from
the previous section to a cohort dataset suitable for survival analysis
and illustrate algorithms to match controls to cases and to cut cohort
data by time-varying covariates. In section 5 we briefly discuss some
accessory functions provided in the package. In the final section we
discuss the \texttt{.ehr} environment used to define the EHR database
being used and how this can be set to work with different databases.

We have provided a number of simulated flat files to demonstrate the
functions provided with the package.

\section{2. Importing EHR data}\label{importing-ehr-data}

rEHR is installed and loaded in the usual way:

\begin{Shaded}
\begin{Highlighting}[]
\NormalTok{if(!}\StringTok{ "rEHR"} \NormalTok\StringTok{ }\KeywordTok{rownames}\NormalTok{(}\KeywordTok{installed.packages}\NormalTok{())) }\KeywordTok{install.packages}\NormalTok{(}\StringTok{"rEHR"}\NormalTok{)}
\KeywordTok{library}\NormalTok{(rEHR)}
\end{Highlighting}
\end{Shaded}

The development version of the package is available from
\href{http://www.github.com}{Github} and is accessible via the devtools
(Wickham and Chang 2014) package:

\begin{Shaded}
\begin{Highlighting}[]
\KeywordTok{library}\NormalTok{(devtools)}
\KeywordTok{install_github}\NormalTok{(}\StringTok{"rOpenHealth/rEHR"}\NormalTok{)}
\KeywordTok{library}\NormalTok{(rEHR)}
\end{Highlighting}
\end{Shaded}

EHR data are stored as relational databases but are most commonly made
available to researchers in the form of flat text files. This has the
advantage of easier access for simple tasks and, for example, viewing
the files in a spreadsheet. However, most non-trivial operations require
researchers to iterate over a series of (potentially large) different
groups of files. For example here we present pseudocode for a simple
workflow leading to the production of a dataset of prevalent cases for a
condition such as diabetes:

\begin{Shaded}
\begin{Highlighting}[]
\CommentTok{# Pseudocode prevalent cases algorithm    }
\NormalTok{define a list of clinical codes for the condition}
\NormalTok{for each practice:}
\StringTok{    }\NormalTok{load clinical events }\KeywordTok{files} \NormalTok{(clinical, referral, drugs etc.)}
    \NormalTok{select clinical events matching the clinical code list}
    \NormalTok{load patient and practice files}
    \NormalTok{for each year:}
\StringTok{        }\NormalTok{select active patients}
        \NormalTok{select events in year}
        \NormalTok{merge active patients and events in year according to condition algorithm}
    \NormalTok{combine all years in practice}
\NormalTok{combine patients in all practices}
\end{Highlighting}
\end{Shaded}

Each level of iteration (represented by the nested \texttt{for loops})
and each type of file (e.g.~clinical, referral, drugs etc.) in the above
algorithm introduces the opportunity for bugs to creep into extraction
code, while the repeated opening and closing of multiple text files,
combined with the inherent inefficiency of for loops in \texttt{R} often
result in slow, error prone code. The \texttt{rEHR} package allows
researchers to first automatically import these flat files into a SQLite
database and then use predefined functions to query this database
efficiently and precisely. We use SQLite databases for a variety of
reasons:

\begin{itemize}
\itemsep1pt\parskip0pt\parsep0pt
\item
  SQLite databases are stored as files in the directory system of the
  computer and require no installation setup. SQLite3 is installed
  automatically as a result of installing the dependencies for the
  package
\item
  SQLite files are stored efficiently and are relatively small compared
  to text files
\item
  The SQL language has been optimised for very rapid and efficient
  queries of SQLite files, resulting in much faster queries than would
  be available to multiple flat files
\item
  Working with SQLite databases allows users to use some very well
  developed tools that are already available to the \texttt{R} community
  such as \texttt{sqldf} (Grothendieck 2014) and \texttt{RSQLite}
  (James, Falcon, and SQLite 2013) if they are familiar with \texttt{R}
  SQL integration tools. These tools also allow for more specific tool
  functions to be built to shield users from the complexities of SQL
  queries.
\end{itemize}

\begin{Shaded}
\begin{Highlighting}[]
\NormalTok{## Use the simulated ehr files supplied with the package to build our database}
\NormalTok{ehr_path <-}\StringTok{ }\KeywordTok{dirname}\NormalTok{(}\KeywordTok{system.file}\NormalTok{(}\StringTok{"ehr_data"}\NormalTok{, }\StringTok{"ehr_Clinical.txt"}\NormalTok{, }\DataTypeTok{package =} \StringTok{"rEHR"}\NormalTok{))}
\NormalTok{## create a new database connection to a temporary file}
\NormalTok{db <-}\StringTok{ }\KeywordTok{database}\NormalTok{(}\KeywordTok{tempfile}\NormalTok{(}\DataTypeTok{fileext =} \StringTok{".sqlite"}\NormalTok{))}
\NormalTok{## Import multiple data files into the database}
\KeywordTok{import_CPRD_data}\NormalTok{(db, }\DataTypeTok{data_dir =} \NormalTok{ehr_path,}
                 \DataTypeTok{filetypes =} \KeywordTok{c}\NormalTok{(}\StringTok{"Clinical"}\NormalTok{, }\StringTok{"Consultation"}\NormalTok{, }
                               \StringTok{"Patient"}\NormalTok{, }\StringTok{"Practice"}\NormalTok{, }
                               \StringTok{"Referral"}\NormalTok{),}
                 \DataTypeTok{dateformat =} \StringTok{"%Y-%m-%d"}\NormalTok{, }
                 \DataTypeTok{yob_origin =} \DecValTok{1800}\NormalTok{,}
                 \DataTypeTok{regex =} \StringTok{"ehr"}\NormalTok{,}
                 \DataTypeTok{recursive =} \OtherTok{TRUE}\NormalTok{)}
\NormalTok{## Individual files can also be added:}
\KeywordTok{add_to_database}\NormalTok{(db, }\DataTypeTok{files =} \KeywordTok{system.file}\NormalTok{(}\StringTok{"ehr_data"}\NormalTok{, }\StringTok{"ehr_Therapy.txt"}\NormalTok{, }\DataTypeTok{package =} \StringTok{"rEHR"}\NormalTok{), }
                \DataTypeTok{table_name =} \StringTok{"Therapy"}\NormalTok{, }\DataTypeTok{dateformat =} \StringTok{"%Y-%m-%d"}\NormalTok{)}
\NormalTok{## Use the overloaded `head` function to view a list of }
\NormalTok{## tables or the head of individual tables:}
\KeywordTok{head}\NormalTok{(db)}
\end{Highlighting}
\end{Shaded}

\begin{verbatim}
##    type         name     tbl_name
## 1 table     Clinical     Clinical
## 2 table Consultation Consultation
## 3 table      Patient      Patient
## 4 table     Practice     Practice
## 5 table     Referral     Referral
## 6 table      Therapy      Therapy
\end{verbatim}

\begin{Shaded}
\begin{Highlighting}[]
\KeywordTok{head}\NormalTok{(db, }\DataTypeTok{table =} \StringTok{"Clinical"}\NormalTok{)}
\end{Highlighting}
\end{Shaded}

\begin{verbatim}
##   patid  eventdate constype consid medcode        comorbidity practid
## 1  1001 2003-08-25        0      4   69753       hypertension       1
## 2  1001 2004-04-13        1      5   96277 atrial_fibrilation       1
## 3  1001 2004-04-13        1      5    2212 atrial_fibrilation       1
## 4  1001 2004-04-13        1      5   96076 atrial_fibrilation       1
## 5  1001 2005-02-08        1      6   23579                chd       1
## 6  1001 2005-02-18        1      7   16059       hypertension       1
\end{verbatim}

The \texttt{import\_CPRD\_data} and \texttt{add\_to\_database} functions
are able to import tab-delimited text files or zipped tab-delimited
text-files. By default, all date strings are converted to R dates with
standard ISO format (``\%Y-\%m-\%d''). A \texttt{regex} argument should
be supplied that is a regular expression to match a common prefix to the
filenames, separated from the file type by an underscore.

\section{3. Querying the database}\label{querying-the-database}

\subsection{Selecting all events}\label{selecting-all-events}

Once EHR data has been imported to the database, the \texttt{rEHR}
package has a number of flexible built-in querying functions for
extracting data. These functions are much faster to execute and less
error prone than having to loop through hundreds of text files.

The primary generic query function is \texttt{select\_events()} and is
able to select all the events in a database table matching a provided
\texttt{where} argument. This function is also called by the other more
specific query functions. An example set of lists of clinical codes for
a number of medical conditions is provided with the package
(\texttt{data(clinical\_codes)}). \texttt{select\_events()} returns a
dataframe of extracted data.

\begin{Shaded}
\begin{Highlighting}[]
\NormalTok{diabetes_codes <-}\StringTok{ }\NormalTok{clinical_codes[clinical_codes$list ==}\StringTok{ "Diabetes"}\NormalTok{,]}
\KeywordTok{select_events}\NormalTok{(db, }\DataTypeTok{tab =} \StringTok{"Clinical"}\NormalTok{, }\DataTypeTok{columns =} \KeywordTok{c}\NormalTok{(}\StringTok{"patid"}\NormalTok{, }\StringTok{"eventdate"}\NormalTok{, }\StringTok{"medcode"}\NormalTok{), }
              \DataTypeTok{where =} \StringTok{"medcode %in% .(diabetes_codes$medcode) & }
\StringTok{                       eventdate < '2006-01-01' & eventdate >= '2005-01-01'"}\NormalTok{)}
\end{Highlighting}
\end{Shaded}

\begin{verbatim}
##   patid  eventdate medcode
## 1  3012 2005-09-30     273
## 2  1037 2005-04-08     277
## 3  1038 2005-05-19     273
## 4  1091 2005-05-27     351
## 5  1091 2005-07-25     351
## 6  1097 2005-03-10     273
\end{verbatim}

The \texttt{where} argument is equivalent to the WHERE clause in SQL, in
that it is used to select subsets of the data table. The user must
supply a string representation of valid \texttt{R} code, which is then
translated to SQL via the \texttt{dplyr::translate\_sql\_q} function.
There are two important caveats to this:

\begin{enumerate}
\def\labelenumi{\arabic{enumi}.}
\itemsep1pt\parskip0pt\parsep0pt
\item
  If an element of the clause represents an R object to be accessed
  (such as the elements of a vector) it must be wrapped in a
  \texttt{.()} (See the example above). String elements wrapped in
  \texttt{.()} are processed by the \texttt{expand\_string} function
  before being passed to \texttt{dplyr::translate\_sql\_q}.
\item
  Dates should separately quoted and entered in ISO format
  (`\%Y-\%m-\%d'). This is because dates are stored as ISO text in the
  database, not as r Date types.
\end{enumerate}

If the argument \texttt{sql\_only == TRUE}, the function only generates
the SQL needed for the query, rather than running the query itself. In
this way, \texttt{select\_events} can be used as the base for more
complex query functions. The results of this function can also then be
passed to \texttt{temp\_table()} to create temporary tables where it is
not desirable to keep large query results in RAM. For example:

\begin{Shaded}
\begin{Highlighting}[]
\NormalTok{Asthma_codes <-}\StringTok{ }\NormalTok{clinical_codes[clinical_codes$list ==}\StringTok{ "Asthma"}\NormalTok{,]}
\NormalTok{q <-}\StringTok{ }\KeywordTok{select_events}\NormalTok{(db, }\DataTypeTok{tab =} \StringTok{"Clinical"}\NormalTok{, }\DataTypeTok{columns =} \KeywordTok{c}\NormalTok{(}\StringTok{"patid"}\NormalTok{, }\StringTok{"eventdate"}\NormalTok{, }\StringTok{"medcode"}\NormalTok{), }
              \DataTypeTok{where =} \StringTok{"medcode %in% .(Asthma_codes$medcode)"}\NormalTok{, }
              \DataTypeTok{sql_only =} \OtherTok{TRUE}\NormalTok{)}
\KeywordTok{temp_table}\NormalTok{(db, }\DataTypeTok{tab_name =} \StringTok{"Asthma"}\NormalTok{, }\DataTypeTok{select_query =} \NormalTok{q)}
\end{Highlighting}
\end{Shaded}

\begin{verbatim}
## Temporary table 'Asthma' created
\end{verbatim}

\begin{Shaded}
\begin{Highlighting}[]
\KeywordTok{head}\NormalTok{(db, }\DataTypeTok{temp =} \OtherTok{TRUE}\NormalTok{)}
\end{Highlighting}
\end{Shaded}

\begin{verbatim}
##    type   name tbl_name
## 1 table Asthma   Asthma
\end{verbatim}

\begin{Shaded}
\begin{Highlighting}[]
\KeywordTok{head}\NormalTok{(db, }\DataTypeTok{table =} \StringTok{"Asthma"}\NormalTok{)}
\end{Highlighting}
\end{Shaded}

\begin{verbatim}
##   patid  eventdate medcode
## 1  1025 2014-04-11    1105
## 2  1035 2012-03-05    1116
## 3  2065 2006-03-20    1095
\end{verbatim}

\subsubsection{Using raw SQL queries}\label{using-raw-sql-queries}

Since EHR data is stored as a standard SQLite database, users can
alternatively make SQL queries to the database using \texttt{sqldf},
which is imported into the namespace on loading of the \texttt{rEHR}
package:

\begin{Shaded}
\begin{Highlighting}[]
\KeywordTok{sqldf}\NormalTok{(}\StringTok{"SELECT patid, practid, gender, yob, deathdate from Patient WHERE }
\StringTok{          deathdate IS NOT NULL LIMIT 6"}\NormalTok{, }
      \DataTypeTok{connection =} \NormalTok{db)}
\end{Highlighting}
\end{Shaded}

\begin{verbatim}
##   patid practid gender  yob  deathdate
## 1  1003       3      0 1983 2001-11-16
## 2  3015      15      1 1995 2000-05-09
## 3  2016      16      1 1959 2002-10-28
## 4  1018      18      0 1992 2009-12-29
## 5  2020      20      1 1956 2002-11-29
## 6  1023      23      0 1983 2013-03-24
\end{verbatim}

There are two methods for including \texttt{R} objects in raw SQL
strings. First, wrapping the string in a call to
\texttt{expand\_string()} allows for the \texttt{.()} notation to be
used as in \texttt{where} arguments to \texttt{select\_events()} based
functions. Alternatively, a helper function, \texttt{wrap\_sql\_query()}
is provided that functions in a similar way to \texttt{base::sprintf}
but formats objects according to SQL syntax. If the result of evaluating
the argument is a vector of length 1, it is inserted as is; if it is a
vector of length \textgreater{} 1, it is wrapped in parentheses and
comma separated.

\begin{Shaded}
\begin{Highlighting}[]
\NormalTok{medcodes1 <-}\StringTok{ }\DecValTok{1}\NormalTok{:}\DecValTok{5}
\NormalTok{practice <-}\StringTok{ }\DecValTok{255}
\KeywordTok{expand_string}\NormalTok{(}\StringTok{"SELECT * FROM clinical WHERE practid == .(practice)"}\NormalTok{)}
\end{Highlighting}
\end{Shaded}

\begin{verbatim}
## [1] "SELECT * FROM clinical WHERE practid == 255"
\end{verbatim}

\begin{Shaded}
\begin{Highlighting}[]
\KeywordTok{wrap_sql_query}\NormalTok{(}\StringTok{"SELECT * FROM clinical WHERE practid == #1 AND medcodes in #2"}\NormalTok{, }
               \NormalTok{practice, medcodes1)}
\end{Highlighting}
\end{Shaded}

\begin{verbatim}
## [1] "SELECT * FROM clinical WHERE practid == 255 AND medcodes in ( 1, 2, 3, 4, 5 )"
\end{verbatim}

\subsection{Selecting first or last
events}\label{selecting-first-or-last-events}

Frequently, users need to find the first clinical event for a given
patient (e.g.~to identify dates of diagnosis of chronic diseases) or the
most recent clinical event (e.g.~to identify if a drug therapy has been
prescribed within a certain time period). \texttt{rEHR} provides
convenience functions for these common situations. The functions run a
\texttt{select\_events()} query and then group by patient id and selects
only the earliest/latest event for each patient:

\begin{Shaded}
\begin{Highlighting}[]
\NormalTok{first_DM <-}\StringTok{ }\KeywordTok{first_events}\NormalTok{(db, }\DataTypeTok{tab =} \StringTok{"Clinical"}\NormalTok{, }
                         \DataTypeTok{columns =} \KeywordTok{c}\NormalTok{(}\StringTok{"patid"}\NormalTok{, }\StringTok{"eventdate"}\NormalTok{, }\StringTok{"medcode"}\NormalTok{), }
              \DataTypeTok{where =} \StringTok{"medcode %in% .(diabetes_codes$medcode)"}\NormalTok{)}
\NormalTok{last_DM <-}\StringTok{ }\KeywordTok{last_events}\NormalTok{(db, }\DataTypeTok{tab =} \StringTok{"Clinical"}\NormalTok{, }
                       \DataTypeTok{columns =} \KeywordTok{c}\NormalTok{(}\StringTok{"patid"}\NormalTok{, }\StringTok{"eventdate"}\NormalTok{, }\StringTok{"medcode"}\NormalTok{), }
              \DataTypeTok{where =} \StringTok{"medcode %in% .(diabetes_codes$medcode)"}\NormalTok{)}
\KeywordTok{head}\NormalTok{(first_DM)}
\end{Highlighting}
\end{Shaded}

\begin{verbatim}
##   patid  eventdate medcode
## 1  1004 2007-12-25     351
## 2  1005 2004-08-31     351
## 3  1008 2002-03-02     351
## 4  1010 2014-04-11     351
## 5  1012 2012-05-28     351
## 6  1015 2008-08-16     351
\end{verbatim}

\begin{Shaded}
\begin{Highlighting}[]
\KeywordTok{head}\NormalTok{(last_DM)}
\end{Highlighting}
\end{Shaded}

\begin{verbatim}
##   patid  eventdate medcode
## 1  1004 2007-12-25     351
## 2  1005 2009-03-09     351
## 3  1008 2002-03-02     351
## 4  1010 2014-04-11     351
## 5  1012 2013-02-14     351
## 6  1015 2013-08-17     273
\end{verbatim}

\subsection{Querying longitudinal data with
\texttt{select\_by\_year()}}\label{querying-longitudinal-data-with-selectux5fbyux5fyear}

Researchers will often want to extract data over a range of different
time-points, for example they may want to calculate the prevalence of a
condition and how this changes through time. When working with flat text
files, this must be done with a complex nested loop that is both slow
and error-prone. The \texttt{select\_by\_year()} function provides a
simple interface to extract longitudinal data. On posix-compliant
computers (Linux, BSD, Mac), this function can make use of parallel
processes to select data for different years concurrently, greatly
accelerating the extraction process on multicore machines. The function
runs a series of selects over a year range and collects in a list of
dataframes.

The function applies a database select over a range of years and outputs
as a list or a dataframe. Either a database object or a path to a
database file can be supplied. If multiple cores are being used
(i.e.~cores \textgreater{} 1), a path to a database file must be used
because the same database connection cannot be used across threads. In
this case, a new database connection is made with every fork. Note that
when working with temporary tables, \texttt{cores} must be set to 1 and
the open database connection must be set with \texttt{db}. This is
because the use of \texttt{parallel::mclapply} means that new database
connections need to be started for each fork and temporary files are
only available inside the same connection.

Queries can be made against multiple tables, assuming that the columns
being extracted are present in all tables. The \texttt{columns} argument
is a character vector of column names to be selected. The individual
elements can be of arbitrary length. This means it is possible to insert
SQL clauses e.g. ``DISTINCT patid''.

A numeric vector of years is passed to the \texttt{year\_range} argument
to specify the years to select data for. Selection is done according to
the function passed to the \texttt{selector\_fn} argument.
\texttt{select\_events} is the default but \texttt{first\_events} and
\texttt{last\_events} can also be used, as well as custom selection
functions. The \texttt{where} argument works in the same way as in
\texttt{select\_events} except that year-start and year-end criteria can
be added as `STARTDATE' and `ENDDATE'. These are translated to the
correct year- start and end dates. Different start and end dates can be
specified by supplying a function to the \texttt{year\_fn} argument.
This function must accept a single year argument and return a list with
two elements - ``startdate'' and ``enddate'', each of which must be date
characters in posix format (i.e. ``\%Y-\%m-\%d''). Three functions are
provided to define years (\texttt{standard\_years} for 1st January to
31st December, \texttt{qof\_years} for UK financial years as used in the
UK Quality and Outcomes Framework (Roland 2004) and
\texttt{qof\_15\_months} for the period starting 1st January in the year
in question and finishing on the 31st March the following year) and a
convenience function, \texttt{build\_date\_fn()} is provided to which
users can supply lists of year offsets, months and days for year- start
and end to return a function that can be supplied as the
\texttt{year\_fn} argument. Finally the user can set the
\texttt{as\_list} argument to determine whether data from each year is
returned as a separate list element or as a single data frame.

\subsubsection{Selecting prevalent and incident
events}\label{selecting-prevalent-and-incident-events}

To show the utility of the package we demonstrate how one might extract
an incident and prevalent cohort of diabetes patients from the simulated
example data. Prevalent events for a chronic condition are selected by
the earliest diagnostic event prior to the end of the time period in
question. The denominator for the calculation of the prevalence is the
total number of patients registered at that time point.

\begin{Shaded}
\begin{Highlighting}[]
\CommentTok{# Select all patients with current registration date (crd) < the start date }
\CommentTok{# for each year.}
\NormalTok{registered_patients <-}\StringTok{ }\KeywordTok{select_by_year}\NormalTok{(}\DataTypeTok{db =} \NormalTok{db, }
                         \DataTypeTok{tables =} \StringTok{"patient"}\NormalTok{, }
                         \DataTypeTok{columns =} \KeywordTok{c}\NormalTok{(}\StringTok{"patid"}\NormalTok{, }\StringTok{"practid"}\NormalTok{, }\StringTok{"gender"}\NormalTok{, }
                                     \StringTok{"yob"}\NormalTok{, }\StringTok{"crd"}\NormalTok{, }\StringTok{"tod"}\NormalTok{, }\StringTok{"deathdate"}\NormalTok{), }
                         \DataTypeTok{where =} \StringTok{"crd < STARTDATE"}\NormalTok{,}
                         \DataTypeTok{year_range =} \KeywordTok{c}\NormalTok{(}\DecValTok{2008}\NormalTok{:}\DecValTok{2012}\NormalTok{), }
                         \DataTypeTok{year_fn =} \NormalTok{standard_years)}
\end{Highlighting}
\end{Shaded}

\begin{verbatim}
## Using open database connection
\end{verbatim}

\begin{Shaded}
\begin{Highlighting}[]
\KeywordTok{str}\NormalTok{(registered_patients)}
\end{Highlighting}
\end{Shaded}

\begin{verbatim}
## Classes 'tbl_df', 'tbl' and 'data.frame':    1005 obs. of  8 variables:
##  $ patid    : int  1001 1002 2002 3002 4002 1003 2003 1004 2004 3004 ...
##  $ practid  : int  1 2 2 2 2 3 3 4 4 4 ...
##  $ gender   : int  1 1 1 1 0 0 1 0 1 1 ...
##  $ yob      : num  1989 1942 1965 1959 1932 ...
##  $ crd      : chr  "1998-03-22" "2003-07-10" "1997-10-15" "1981-09-01" ...
##  $ tod      : chr  NA NA NA NA ...
##  $ deathdate: chr  NA NA NA NA ...
##  $ year     : int  2008 2008 2008 2008 2008 2008 2008 2008 2008 2008 ...
\end{verbatim}

\begin{Shaded}
\begin{Highlighting}[]
\KeywordTok{table}\NormalTok{(registered_patients$year)}
\end{Highlighting}
\end{Shaded}

\begin{verbatim}
## 
## 2008 2009 2010 2011 2012 
##  189  195  201  206  214
\end{verbatim}

Notice that \texttt{select\_by\_year} returns a dataframe in long form,
with a year column for the longitudinal component. Next we calculate the
incident cases, which are those patients with first diagnoses at any
point before the end of the year in question, plus the dates for the
first diagnoses. In this case we include events matching our list of
diabetes clinical codes in either clinical or referral files. Because we
only want the first diagnosis dates we set the \texttt{selector\_fn}
argument to \texttt{first\_events}:

\begin{Shaded}
\begin{Highlighting}[]
\NormalTok{incident_cases <-}\StringTok{ }\KeywordTok{select_by_year}\NormalTok{(}\DataTypeTok{db =} \NormalTok{db, }
                                      \DataTypeTok{tables =} \KeywordTok{c}\NormalTok{(}\StringTok{"Clinical"}\NormalTok{, }\StringTok{"Referral"}\NormalTok{), }
                                      \DataTypeTok{columns =} \KeywordTok{c}\NormalTok{(}\StringTok{"patid"}\NormalTok{, }\StringTok{"eventdate"}\NormalTok{, }\StringTok{"medcode"}\NormalTok{), }
                                      \DataTypeTok{where =} \StringTok{"medcode %in% .(diabetes_codes$medcode) & }
\StringTok{                                               eventdate <= ENDDATE"}\NormalTok{,}
                                      \DataTypeTok{year_range =} \KeywordTok{c}\NormalTok{(}\DecValTok{2008}\NormalTok{:}\DecValTok{2012}\NormalTok{), }
                                      \DataTypeTok{year_fn =} \NormalTok{standard_years, }
                                      \DataTypeTok{selector_fn =} \NormalTok{first_events)}
\end{Highlighting}
\end{Shaded}

\begin{verbatim}
## Using open database connection
\end{verbatim}

\begin{Shaded}
\begin{Highlighting}[]
\KeywordTok{str}\NormalTok{(incident_cases)}
\end{Highlighting}
\end{Shaded}

\begin{verbatim}
## Classes 'tbl_df', 'tbl' and 'data.frame':    262 obs. of  5 variables:
##  $ patid    : int  1004 1005 1008 1015 1025 1035 1037 1038 1043 1047 ...
##  $ eventdate: chr  "2007-12-25" "2004-08-31" "2002-03-02" "2008-08-16" ...
##  $ medcode  : int  351 351 351 351 351 293 277 273 351 257 ...
##  $ table    : chr  "Clinical" "Clinical" "Clinical" "Clinical" ...
##  $ year     : int  2008 2008 2008 2008 2008 2008 2008 2008 2008 2008 ...
\end{verbatim}

Note that in this case extra columns have been added for both year and
table, to identify the table the event was found in. Because events were
taken from more than one table (Clinical and Referrals), the
incident\_cases dataframe should be sorted and duplicates removed to
ensure that only the first events are kept. The two datasets are then
merged to give the dataset from which the denominators and numerators
can be calculated. The \texttt{dplyr} package is imported to the
namespace when the \texttt{rEHR} package is loaded. This simplifies and
accelerates merging operations and is an important part of the
\texttt{rEHR} workflow:

\begin{Shaded}
\begin{Highlighting}[]
\NormalTok{## Remove duplicates across clinical and referral tables:}
\NormalTok{incident_cases %>%}
\StringTok{    }\KeywordTok{group_by}\NormalTok{(patid, year) %>%}
\StringTok{    }\KeywordTok{arrange}\NormalTok{(eventdate) %>%}
\StringTok{    }\KeywordTok{distinct}\NormalTok{() %>%}
\StringTok{    }\NormalTok{ungroup ->}\StringTok{ }\NormalTok{incident_cases}
\NormalTok{## All patients are kept (equivalent to merge(all.x = TRUE))}
\NormalTok{prevalence_dat <-}\StringTok{ }\KeywordTok{left_join}\NormalTok{(registered_patients, incident_cases)}
\end{Highlighting}
\end{Shaded}

Prevalence and incidence can be calculated by the built-in functions
\texttt{prev\_terms()} and \texttt{prev\_totals()}.
\texttt{prev\_terms()} adds logical columns for membership of incidence
and prevalence denominators as well as a column for the contribution of
the individual to that year's followup time. \texttt{prev\_totals()}
summarises this information to calculate the denominators and numerators
for prevalence and incidence, according to the users' grouping factors.
The criteria for membership of the incidence and prevalence numerators
and denominators as well as for followup time are shown in table 1.

\begin{longtable}[c]{@{}ll@{}}
\toprule
Column & Definition\tabularnewline
\midrule
\endhead
Incident Numerator & existing event date + event occurs within year +
transfer out date \textgreater{} event date\tabularnewline
Incident Denominator & No events in previous years + transfer out date
\textgreater{} year start date\tabularnewline
Prevalent Numerator & existing event date + transfer out date
\textgreater{} event date\tabularnewline
Prevalent Denominator & transfer out date \textgreater{} year start
date\tabularnewline
Followup & minimum of (year end date, transfer out date, death date) -
year start date\tabularnewline
\bottomrule
\end{longtable}

table 1: Definitions of incidence and prevalence terms

\begin{Shaded}
\begin{Highlighting}[]
\NormalTok{prevalence_dat <-}\StringTok{ }\KeywordTok{prev_terms}\NormalTok{(prevalence_dat)}
\NormalTok{totals <-}\StringTok{ }\KeywordTok{prev_totals}\NormalTok{(prevalence_dat)}
\NormalTok{totals$prevalence$year_counts}
\end{Highlighting}
\end{Shaded}

\begin{verbatim}
## Source: local data frame [5 x 4]
## 
##   year numerator denominator prevalence
## 1 2008        31    174.6721   17.74754
## 2 2009        35    179.3785   19.51181
## 3 2010        41    183.1403   22.38721
## 4 2011        50    185.4182   26.96607
## 5 2012        55    191.5838   28.70806
\end{verbatim}

\begin{Shaded}
\begin{Highlighting}[]
\NormalTok{totals$incidence$year_counts}
\end{Highlighting}
\end{Shaded}

\begin{verbatim}
## Source: local data frame [5 x 4]
## 
##   year numerator denominator incidence
## 1 2008         4    143.9014  2.779680
## 2 2009         3    144.4983  2.076149
## 3 2010         4    142.2806  2.811345
## 4 2011         7    135.5893  5.162648
## 5 2012         5    137.4675  3.637224
\end{verbatim}

Here we see that, in our simulated dataset, we have a diabetes
prevalence of 17.7\% in 2008 raising to 28.7\% in 2012 and an incidence
of 2.8\% in 2008 dropping to 3.6\% in 2012.

\section{4. Building cohorts, matching and time-varying
covariates}\label{building-cohorts-matching-and-time-varying-covariates}

In this section we demonstrate how to convert the longitudinal data from
the previous section to a cohort dataset suitable for survival analysis
and also illustrate algorithms to match controls to cases and to cut
cohort data by time-varying covariates.

One of the most common uses of EHR data in research is to build cohorts
for survival analyses. The longitudinal data in the previous section is
easily converted to survival cohort format using the
\texttt{build\_cohort()} function. This returns a dataset with a single
row for each patient and includes only patients in the numerator or
denominator for whichever cohort type is chosen (either incident or
prevalent cohorts). Columns are added for start and end dates and for
start and end times as integer differences from the cohort start date. A
binary column is added to indicate membership of the case group. All
patients with start dates greater than their end dates are removed from
the dataset. The diagnosis\_start argument is used to include the
diagnosis date in the definition of the start dates for the patients. If
it is not required for the diagnosis date to be included in the start
date definition, this argument can be set to \texttt{NULL}. Here, we
will first merge in practice data (i.e.~dates for when practices are
deemed to be up to standard) and then construct the cohort:

\begin{Shaded}
\begin{Highlighting}[]
\NormalTok{practices <-}\StringTok{ }\KeywordTok{select_events}\NormalTok{(}\DataTypeTok{db =} \NormalTok{db, }\DataTypeTok{tab =} \StringTok{"Practice"}\NormalTok{, }\DataTypeTok{convert_dates =} \OtherTok{TRUE}\NormalTok{)}
\NormalTok{prevalence_dat <-}\StringTok{ }\KeywordTok{left_join}\NormalTok{(prevalence_dat, practices)}

\NormalTok{cohort <-}\StringTok{ }\KeywordTok{build_cohort}\NormalTok{(prevalence_dat, }\DataTypeTok{cohort_type =} \StringTok{"prev"}\NormalTok{, }
                       \DataTypeTok{cohort_start =} \StringTok{"2006-01-01"}\NormalTok{, }\DataTypeTok{cohort_end =} \StringTok{"2012-12-31"}\NormalTok{, }
                       \DataTypeTok{diagnosis_start =} \StringTok{"eventdate"}\NormalTok{)}
\end{Highlighting}
\end{Shaded}

The cohort is now ready for analysis. e.g.

\begin{Shaded}
\begin{Highlighting}[]
\NormalTok{## Add a logical column for death during cohort}
\NormalTok{cohort$death <-}\StringTok{ }\KeywordTok{with}\NormalTok{(cohort, }
                     \KeywordTok{ifelse}\NormalTok{(!}\KeywordTok{is.null}\NormalTok{(deathdate) &}\StringTok{ }
\StringTok{                                }\NormalTok{(deathdate >}\StringTok{ }\KeywordTok{as.Date}\NormalTok{(}\StringTok{"2006-01-01"}\NormalTok{) &}\StringTok{ }
\StringTok{                                     }\NormalTok{deathdate <}\StringTok{ }\KeywordTok{as.Date}\NormalTok{(}\StringTok{"2012-12-31"}\NormalTok{)), }
                            \DecValTok{1}\NormalTok{, }\DecValTok{0}\NormalTok{))}
\NormalTok{cohort$death[}\KeywordTok{is.na}\NormalTok{(cohort$death)] <-}\StringTok{ }\DecValTok{0}

\KeywordTok{library}\NormalTok{(survival)}
\NormalTok{surv_obj <-}\StringTok{ }\KeywordTok{with}\NormalTok{(cohort, }\KeywordTok{Surv}\NormalTok{(start, end, death))}
\KeywordTok{coxph}\NormalTok{(surv_obj ~}\StringTok{ }\NormalTok{gender +}\StringTok{ }\NormalTok{case, }\DataTypeTok{data =} \NormalTok{cohort)}
\end{Highlighting}
\end{Shaded}

\begin{verbatim}
## Call:
## coxph(formula = surv_obj ~ gender + case, data = cohort)
## 
## 
##          coef exp(coef) se(coef)      z    p
## gender  0.506     1.659    0.837  0.605 0.55
## case   -0.645     0.524    1.081 -0.597 0.55
## 
## Likelihood ratio test=0.81  on 2 df, p=0.667  n= 199, number of events= 7
\end{verbatim}

\subsection{Matching}\label{matching}

Matching cases to controls is an important pre-analysis step. The
\texttt{rEHR} package provided three methods for matching cases to
controls:

\begin{enumerate}
\def\labelenumi{\arabic{enumi}.}
\itemsep1pt\parskip0pt\parsep0pt
\item
  Incidence density matching (IDM)
\item
  Exact matching
\item
  Matching on a dummy index date sourced from consultation files
\end{enumerate}

\subsubsection{Incidence density
matching}\label{incidence-density-matching}

This is performed using the \texttt{get\_matches()} function. With IDM,
controls are selected for a particular case at the time of diagnosis (or
other event such as death) from from other members of the cohort who, at
that time, do not have the diagnosis. The IDM sampling procedure allows
the same patient to be selected as a control for more than one case,
thus providing a full set controls for each case while still producing
unbiased estimates of risk (Richardson 2004; Reeves et al. 2014). This
also means that the matching procedure can be parallelised to increase
computational efficiency.

\begin{Shaded}
\begin{Highlighting}[]
\NormalTok{cohort2 <-}\StringTok{ }\KeywordTok{build_cohort}\NormalTok{(prevalence_dat, }\DataTypeTok{cohort_type =} \StringTok{"incid"}\NormalTok{, }
                        \DataTypeTok{cohort_start =} \StringTok{"2006-01-01"}\NormalTok{, }\DataTypeTok{cohort_end =} \StringTok{"2012-12-31"}\NormalTok{, }
                        \DataTypeTok{diagnosis_start =} \StringTok{"eventdate"}\NormalTok{)}
\NormalTok{IDM_controls <-}\StringTok{ }\KeywordTok{get_matches}\NormalTok{(}\DataTypeTok{cases =} \KeywordTok{filter}\NormalTok{(cohort2, case ==}\StringTok{ }\DecValTok{1}\NormalTok{), }
                            \DataTypeTok{control_pool =} \KeywordTok{filter}\NormalTok{(cohort2, case ==}\StringTok{ }\DecValTok{0}\NormalTok{), }
                            \DataTypeTok{match_vars =} \KeywordTok{c}\NormalTok{(}\StringTok{"gender"}\NormalTok{, }\StringTok{"region"}\NormalTok{),}
                            \DataTypeTok{n_controls =} \DecValTok{4}\NormalTok{, }\DataTypeTok{cores =} \DecValTok{2}\NormalTok{, }
                            \DataTypeTok{method =} \StringTok{"incidence_density"}\NormalTok{, }\DataTypeTok{diagnosis_date  =} \StringTok{"eventdate"}\NormalTok{)}
\end{Highlighting}
\end{Shaded}

In this example matching scenario, 92 controls were matched to 23 cases,
which is 4 controls matched to each case.

In all of the matching algorithms, matching is performed by default on
categories selected in the \texttt{match\_vars} argument. However, more
complex matching strategies can also be employed via the
\texttt{extra\_conditions} argument. You can wrap calls to expressions
in dotted brackets to automatically expand them. This is particularly
useful when you want to find the value for each individual case. Each
case is denoted by \texttt{CASE}, e.g.
\texttt{"start\_date \textless{} .(CASE\$start\_date)"} will ensure the
start date for controls is prior to the start date for the matched case.
The following code also selects controls whose birth year (\texttt{yob})
is within 2 years either side of their matched case:

\begin{Shaded}
\begin{Highlighting}[]
\NormalTok{IDM_controls2 <-}\StringTok{ }\KeywordTok{get_matches}\NormalTok{(}\DataTypeTok{cases =} \KeywordTok{filter}\NormalTok{(cohort2, case ==}\StringTok{ }\DecValTok{1}\NormalTok{), }
                             \DataTypeTok{control_pool =} \KeywordTok{filter}\NormalTok{(cohort2, case ==}\StringTok{ }\DecValTok{0}\NormalTok{), }
                             \DataTypeTok{match_vars =} \KeywordTok{c}\NormalTok{(}\StringTok{"gender"}\NormalTok{, }\StringTok{"region"}\NormalTok{),}
                             \DataTypeTok{extra_conditions =} \StringTok{"yob >= ( .(CASE$yob) - 2) & }
\StringTok{                             yob <= ( .(CASE$yob) + 2)"}\NormalTok{,}
                             \DataTypeTok{n_controls =} \DecValTok{4}\NormalTok{, }\DataTypeTok{cores =} \DecValTok{2}\NormalTok{, }
                             \DataTypeTok{method =} \StringTok{"incidence_density"}\NormalTok{, }\DataTypeTok{diagnosis_date  =} \StringTok{"eventdate"}\NormalTok{)}
\end{Highlighting}
\end{Shaded}

\subsubsection{Exact matching}\label{exact-matching}

Exact matching only matches controls from the control pool, unlike in
IDM matching. Also, matched controls are removed from the control pool
after each case has been matched, so each control can be used a maximum
of one time. Therefore it is possible to have fewer matched controls for
some cases than are requested via the \texttt{n\_controls} argument.
Because the control pool is being altered for every case, exact matching
is not thread safe and so will only run on a single core. The
\texttt{cores} and \texttt{diagnosis\_date} arguments are ignored when
this method is selected.

\begin{Shaded}
\begin{Highlighting}[]
\NormalTok{exact_controls3 <-}\StringTok{ }\KeywordTok{get_matches}\NormalTok{(}\DataTypeTok{cases =} \KeywordTok{filter}\NormalTok{(cohort2, case ==}\StringTok{ }\DecValTok{1}\NormalTok{), }
                            \DataTypeTok{control_pool =} \KeywordTok{filter}\NormalTok{(cohort2, case ==}\StringTok{ }\DecValTok{0}\NormalTok{), }
                            \DataTypeTok{match_vars =} \KeywordTok{c}\NormalTok{(}\StringTok{"gender"}\NormalTok{, }\StringTok{"region"}\NormalTok{),}
                            \DataTypeTok{n_controls =} \DecValTok{4}\NormalTok{, }\DataTypeTok{cores =} \DecValTok{2}\NormalTok{, }
                            \DataTypeTok{method =} \StringTok{"exact"}\NormalTok{, }\DataTypeTok{diagnosis_date  =} \StringTok{"eventdate"}\NormalTok{)}
\end{Highlighting}
\end{Shaded}

In a small cohort, this can rapidly reduce the control pool, leading to
many cases without matches. In this example, 19 out of 23 were matched
with mean 3.6 controls matched to every case.

\subsubsection{Matching on a dummy index
date}\label{matching-on-a-dummy-index-date}

A common matching approach is to match on an index date, for example the
diagnosis date of the cases or the date followup starts. There are
several reasons to match on index date:

\begin{enumerate}
\def\labelenumi{\arabic{enumi}.}
\itemsep1pt\parskip0pt\parsep0pt
\item
  It ensures cases and controls are followed-up, on average, for the
  same amount of time. Not including an index date for controls may
  result in them being, on average, in the cohort for longer than the
  cases because their cohort start date is not constrained by the index
  date
\item
  There is a possible reduction of detection bias, for example if cases
  are expected to visit their doctors more often because they have more
  co-morbidities
\item
  If controls are known to have attended their practice at around the
  same time as their matched case, it is likely they will experience
  similar conditions in terms of practice policy and active GPs
\item
  Patients who, though registered, have no records of contact with the
  medical system (``Ghost patients'') are excluded
\end{enumerate}

However, the controls will often not have the same index to match on
(this is true by definition if the diagnosis date is used). In this
situation, it is common to match on a dummy index date which may be a
clinical event or interaction in the control's electronic health record
that occurs around the same time as the index date of the case (Parisi
et al. 2015; Gelfand et al. 2006). The \texttt{match\_on\_index()}
function allows for matching on an arbitrary number of categorical
\texttt{match\_var} variables and on continuous variables via the
\texttt{extra\_conditions} argument in the same way as the
\texttt{get\_matches()} function above. In addition, a supplied index
date for each case is matched to event dates in a series of consultation
files (1 file for each practice), providing a dummy index date for
controls of a consultation date within \texttt{index\_diff\_limit} days
of the matched case's index date.

Note that the consultation files must be in flat-file format, i.e.~not
as part of the database, but as text (or other filetype, e.g stata dta)
files. This is the data format provided by CPRD (``Clinical Practice
Research Datalink (CPRD) GOLD''). Although in most situations it is more
efficient to process EHR data in SQL databases, as in the earlier
functions described here, consultation tables are often very large and
searching these for every case in a large cohort would be very slow. By
processing consultation files that have been split by practice, it is
possible to search for matches a practice at a time which is both
efficient and allows for parallel processing to speed the process up
still further. For convenience, a function \texttt{flat\_files()} is
provided that can export a database table to flat files split by
practice in a format of their choosing. The \texttt{match\_on\_index()}
function has an \texttt{import\_fn} argument to use different file
formats (e.g. \texttt{foreign::read.dta} or
\texttt{readstata13::read.dta13} for Stata 12 or Stata 13 file).

\begin{Shaded}
\begin{Highlighting}[]
\NormalTok{consultation_dir <-}\StringTok{ "~/R/rEHR_testing"}
\KeywordTok{flat_files}\NormalTok{(db, }\DataTypeTok{out_dir =} \NormalTok{consultation_dir, }\DataTypeTok{file_type =} \StringTok{"csv"}\NormalTok{)}
\NormalTok{index_controls <-}\StringTok{ }\KeywordTok{match_on_index}\NormalTok{(}\DataTypeTok{cases =} \KeywordTok{filter}\NormalTok{(cohort2, case ==}\StringTok{ }\DecValTok{1}\NormalTok{), }
                                 \DataTypeTok{control_pool =} \KeywordTok{filter}\NormalTok{(cohort2, case ==}\StringTok{ }\DecValTok{0}\NormalTok{),}
                                 \DataTypeTok{index_var =} \StringTok{"eventdate"}\NormalTok{, }
                                 \DataTypeTok{match_vars =} \KeywordTok{c}\NormalTok{(}\StringTok{"gender"}\NormalTok{, }\StringTok{"region"}\NormalTok{),}
                                 \DataTypeTok{index_diff_limit =} \DecValTok{90}\NormalTok{, }
                                 \DataTypeTok{consult_path =} \NormalTok{consultation_dir,}
                                 \DataTypeTok{n_controls =} \DecValTok{4}\NormalTok{,}
                                 \DataTypeTok{import_fn =} \NormalTok{function(x) }\KeywordTok{convert_dates}\NormalTok{(}\KeywordTok{read.csv}\NormalTok{(x)))}
\KeywordTok{unlink}\NormalTok{(consultation_dir, }\DataTypeTok{recursive =} \OtherTok{TRUE}\NormalTok{) }\CommentTok{# clean up constructed dirs after analysis}
\end{Highlighting}
\end{Shaded}

This function performs matching that is still more conservative than the
previous methods, since it requires matching of patients within the same
practice and with consultation dates near the index date. In the test
example above, no matched controls were found which is not surprising
with a control pool of only 143. In practice this method is only
appropriate where there is a control pool of hundreds of thousands or
even millions of patients. If too few controls are found, the constraint
can be relaxed by setting a higher \texttt{index\_diff\_limit}. Setting
this to an arbitrarily high value effectively means that matching is not
done on index date, but just on practice and the other user-specified
matching variables. Users may find that this is a more efficient way to
perform exact matching than using the \texttt{get\_matches()} function.
We have used this method to accelerate matching runs with several
million controls that previously took days or weeks to minutes or a few
hours.

\subsection{Time-varying covariates}\label{time-varying-covariates}

Often, researchers want to cut a survival cohort by time-varying
covariates. In this situation, individual patients may run over more
than one row in the cohort dataset. For example, a drug exposure may
occur after the entry into the cohort and one might be interested in how
this might affect the outcome. In this situation, it is useful to have a
pre-exposure and post-exposure time period in the dataset.

The \texttt{cut\_tv()} function cuts up a dataset based on times
supplied for the time-varying covariate. If there is already a variable
for the time-varying covariate, you can chose to flip the existing
values or increment them. This means the function can be called multiple
times to, e.g.~deal with drugs starting and stopping and also to model
the progression of treatment. Other packages implement similar functions
(e.g.~the \texttt{cutLexis} function from the \texttt{Epi} package
(Bendix Carstensen and Hills 2014)). The \texttt{cut\_tv()} function is
considerably faster than other cutting methods (particularly on large
datasetss), does not require conversion of the dataset to other formats
(such as \texttt{Lexis}), can be parallelised on posix compliant
machines and is designed to be chained with \texttt{dplyr} workflows
using the \texttt{\%\textgreater{}\%} operator. \texttt{cut\_tv()} can
deal with the following scenarios:

\begin{itemize}
\itemsep1pt\parskip0pt\parsep0pt
\item
  \textbf{Binary chronic covariates} e.g.~The time of diagnosis for a
  chronic (unresolvable) condition. This requires a single column
  variable of times from entry in the dataset
\item
  \textbf{Binary covariates} e.g.~times of starting and stopping
  medication. This requires more than one column variable in the
  dataset, one for each start or stop event. The state flips with each
  new change.
\item
  \textbf{Incremental time-varying covariates} e.g.~different stages of
  a condition. This requires a single column variable for each
  incremental stage
\item
  \textbf{Any combination of the above} This is achieved by chaining
  multiple calls together
\end{itemize}

One must supply a dataframe, variable names for entry and exit times,
the time-varying covariate, the patient id and the constructed variable.
Also one supplies the number of processor cores to run the function on
and the behaviour of the function if the constructed variable already
exists (either to flip from 1-0 or to increment by one). Here we
demonstrate the different scenarios with a small sample dataset:

\begin{Shaded}
\begin{Highlighting}[]
\NormalTok{tv_test <-}\StringTok{ }\KeywordTok{data.frame}\NormalTok{(}\DataTypeTok{id =} \DecValTok{1}\NormalTok{:}\DecValTok{5}\NormalTok{, }\DataTypeTok{start =} \KeywordTok{rep}\NormalTok{(}\DecValTok{0}\NormalTok{, }\DecValTok{5}\NormalTok{), }\DataTypeTok{end =} \KeywordTok{c}\NormalTok{(}\DecValTok{1000}\NormalTok{, }\DecValTok{689}\NormalTok{, }\DecValTok{1000}\NormalTok{, }\DecValTok{874}\NormalTok{, }\DecValTok{777}\NormalTok{), }
                   \DataTypeTok{event =} \KeywordTok{c}\NormalTok{(}\DecValTok{0}\NormalTok{,}\DecValTok{1}\NormalTok{,}\DecValTok{0}\NormalTok{,}\DecValTok{1}\NormalTok{,}\DecValTok{1}\NormalTok{), }\DataTypeTok{drug_1 =} \KeywordTok{c}\NormalTok{(}\OtherTok{NA}\NormalTok{, }\OtherTok{NA}\NormalTok{, }\OtherTok{NA}\NormalTok{, }\DecValTok{340}\NormalTok{, }\DecValTok{460}\NormalTok{),}
                   \DataTypeTok{drug_2 =} \KeywordTok{c}\NormalTok{(}\OtherTok{NA}\NormalTok{, }\DecValTok{234}\NormalTok{, }\DecValTok{554}\NormalTok{, }\DecValTok{123}\NormalTok{, }\OtherTok{NA}\NormalTok{), }
                   \DataTypeTok{drug_3_start =} \KeywordTok{c}\NormalTok{(}\DecValTok{110}\NormalTok{, }\DecValTok{110}\NormalTok{,}\DecValTok{111}\NormalTok{, }\DecValTok{109}\NormalTok{, }\DecValTok{110}\NormalTok{),}
                   \DataTypeTok{drug_3_stop =} \KeywordTok{c}\NormalTok{(}\DecValTok{400}\NormalTok{, }\DecValTok{400}\NormalTok{, }\DecValTok{400}\NormalTok{, }\DecValTok{400}\NormalTok{, }\DecValTok{400}\NormalTok{),}
                   \DataTypeTok{stage_1 =} \KeywordTok{c}\NormalTok{(}\DecValTok{300}\NormalTok{, }\OtherTok{NA}\NormalTok{, }\OtherTok{NA}\NormalTok{, }\OtherTok{NA}\NormalTok{, }\OtherTok{NA}\NormalTok{),}
                      \DataTypeTok{stage_2 =} \KeywordTok{c}\NormalTok{(}\DecValTok{450}\NormalTok{, }\OtherTok{NA}\NormalTok{, }\OtherTok{NA}\NormalTok{, }\OtherTok{NA}\NormalTok{, }\OtherTok{NA}\NormalTok{))}

\NormalTok{## Multiple binary chronic covariates:}
\NormalTok{tv_out1 <-}\StringTok{ }\KeywordTok{cut_tv}\NormalTok{(tv_test, }
                  \DataTypeTok{entry =} \NormalTok{start, }
                  \DataTypeTok{exit =}  \NormalTok{end, }
                  \DataTypeTok{cut_var =} \NormalTok{drug_1, }
                  \DataTypeTok{id_var =} \NormalTok{id, }
                  \DataTypeTok{tv_name =} \NormalTok{drug_1_state)}
\NormalTok{tv_out1 <-}\StringTok{ }\KeywordTok{cut_tv}\NormalTok{(tv_out1, start, end, drug_2, }\DataTypeTok{id_var =} \NormalTok{id, drug_2_state)}
\KeywordTok{head}\NormalTok{(tv_out1)}
\end{Highlighting}
\end{Shaded}

\begin{verbatim}
## Source: local data frame [6 x 12]
## 
##   id start  end event drug_1 drug_2 drug_3_start drug_3_stop stage_1
## 1  1     0 1000     0     NA     NA          110         400     300
## 2  2     0  233     1     NA    234          110         400      NA
## 3  2   234  689     1     NA    234          110         400      NA
## 4  3     0  553     0     NA    554          111         400      NA
## 5  3   554 1000     0     NA    554          111         400      NA
## 6  4     0  122     1    340    123          109         400      NA
## Variables not shown: stage_2 (dbl), drug_1_state (dbl), drug_2_state (dbl)
\end{verbatim}

\begin{Shaded}
\begin{Highlighting}[]
\NormalTok{## Binary covariates:}
\NormalTok{tv_out3 <-}\StringTok{ }\KeywordTok{cut_tv}\NormalTok{(tv_test, start, end, drug_3_start, }\DataTypeTok{id_var =} \NormalTok{id, drug_3_state)}
\NormalTok{tv_out3 <-}\StringTok{ }\KeywordTok{cut_tv}\NormalTok{(tv_out3, start, end, drug_3_stop, }\DataTypeTok{id_var =} \NormalTok{id, drug_3_state)}
\KeywordTok{head}\NormalTok{(tv_out3)}
\end{Highlighting}
\end{Shaded}

\begin{verbatim}
## Source: local data frame [6 x 11]
## 
##   id start  end event drug_1 drug_2 drug_3_start drug_3_stop stage_1
## 1  1     0  109     0     NA     NA          110         400     300
## 2  1   110  399     0     NA     NA          110         400     300
## 3  1   400 1000     0     NA     NA          110         400     300
## 4  2     0  109     1     NA    234          110         400      NA
## 5  2   110  399     1     NA    234          110         400      NA
## 6  2   400  689     1     NA    234          110         400      NA
## Variables not shown: stage_2 (dbl), drug_3_state (dbl)
\end{verbatim}

\begin{Shaded}
\begin{Highlighting}[]
\NormalTok{## incremental covariates:}
\NormalTok{inc_1 <-}\StringTok{ }\KeywordTok{cut_tv}\NormalTok{(tv_test, start, end, stage_1, }\DataTypeTok{id_var =} \NormalTok{id, disease_stage, }
                \DataTypeTok{on_existing =} \StringTok{"inc"}\NormalTok{)}
\NormalTok{inc_1 <-}\StringTok{ }\KeywordTok{cut_tv}\NormalTok{(inc_1, start, end, stage_2, }\DataTypeTok{id_var =} \NormalTok{id, disease_stage, }
                \DataTypeTok{on_existing =} \StringTok{"inc"}\NormalTok{)}
\KeywordTok{head}\NormalTok{(inc_1)}
\end{Highlighting}
\end{Shaded}

\begin{verbatim}
## Source: local data frame [6 x 11]
## 
##   id start  end event drug_1 drug_2 drug_3_start drug_3_stop stage_1
## 1  1     0  299     0     NA     NA          110         400     300
## 2  1   300  449     0     NA     NA          110         400     300
## 3  1   450 1000     0     NA     NA          110         400     300
## 4  2     0  689     1     NA    234          110         400      NA
## 5  3     0 1000     0     NA    554          111         400      NA
## 6  4     0  874     1    340    123          109         400      NA
## Variables not shown: stage_2 (dbl), disease_stage (dbl)
\end{verbatim}

\begin{Shaded}
\begin{Highlighting}[]
\NormalTok{## Chaining combinations of the above using %>%}
\KeywordTok{library}\NormalTok{(dplyr)}
\NormalTok{tv_test %>%}
\StringTok{    }\KeywordTok{cut_tv}\NormalTok{(start, end, drug_1, }\DataTypeTok{id_var =} \NormalTok{id, drug_1_state) %>%}\StringTok{ }
\StringTok{    }\KeywordTok{cut_tv}\NormalTok{(start, end, drug_2, }\DataTypeTok{id_var =} \NormalTok{id, drug_2_state) %>%}
\StringTok{    }\KeywordTok{cut_tv}\NormalTok{(start, end, drug_3_start, }\DataTypeTok{id_var =} \NormalTok{id, drug_3_state) %>%}
\StringTok{    }\KeywordTok{cut_tv}\NormalTok{(start, end, drug_3_stop, }\DataTypeTok{id_var =} \NormalTok{id, drug_3_state) %>%}
\StringTok{    }\KeywordTok{cut_tv}\NormalTok{(start, end, stage_1, }\DataTypeTok{id_var =} \NormalTok{id, disease_stage, }\DataTypeTok{on_existing =} \StringTok{"inc"}\NormalTok{) %>%}
\StringTok{    }\KeywordTok{cut_tv}\NormalTok{(start, end, stage_2, }\DataTypeTok{id_var =} \NormalTok{id, disease_stage, }\DataTypeTok{on_existing =} \StringTok{"inc"}\NormalTok{) %>%}
\StringTok{    }\NormalTok{head}
\end{Highlighting}
\end{Shaded}

\begin{verbatim}
## Source: local data frame [6 x 14]
## 
##   id start  end event drug_1 drug_2 drug_3_start drug_3_stop stage_1
## 1  1     0  109     0     NA     NA          110         400     300
## 2  1   110  299     0     NA     NA          110         400     300
## 3  1   300  399     0     NA     NA          110         400     300
## 4  1   400  449     0     NA     NA          110         400     300
## 5  1   450 1000     0     NA     NA          110         400     300
## 6  2     0  109     1     NA    234          110         400      NA
## Variables not shown: stage_2 (dbl), drug_1_state (dbl), drug_2_state
##   (dbl), drug_3_state (dbl), disease_stage (dbl)
\end{verbatim}

\section{5. Accessory functions}\label{accessory-functions}

In this section we briefly discuss some miscellaneous functions provided
in the package.

\subsection{Clinical code list
construction}\label{clinical-code-list-construction}

An important part of EHR analyses is the construction of lists of
clinical codes to define conditions, comorbidities and other clinical
entities of interest to the study (D. A. Springate et al. (2014)). We
have previously described methodologies to construct draft lists of
clinical codes from keyword and code searches (Olier et al. (2015)). The
\texttt{R} implementation of this methodology is now part of the rEHR
package.

Building draft lists of clinical codes is a two-stage process: First,
the search is defined by instantiating an object of class
\texttt{MedicalDefinition}, containing the terms to be searched for in
the lookup tables. \texttt{MedicalDefinition} objects can be
instantiated from terms defined within \texttt{R} or imported from a csv
file. The constructor function can be provided with lists of:
\texttt{terms}(clinical search terms), \texttt{codes} (clinical codes),
\texttt{tests} (test search terms), \texttt{drugs} (drug search terms),
\texttt{drugcodes} (drug product codes). Within the individual argument
lists, vectors of length \textgreater{} 1 are searched for together
(logical AND), in any order. Different vectors in the same list are
searched for separately (logical OR). Placing a ``-'' character at the
start of a character vector element excludes that terms from the search.
Providing \texttt{NULL} to any of the arguments means that this element
will not be searched for. Underscores are treated as spaces. When
searching for codes, a range of clinical codes can be searched for by
providing two codes separated by a hyphen. e.g ``E114-E117z''.

\begin{Shaded}
\begin{Highlighting}[]
\NormalTok{## Example construction of a clinical code list}
\NormalTok{def <-}\StringTok{ }\KeywordTok{MedicalDefinition}\NormalTok{(}
    \DataTypeTok{terms =} \KeywordTok{list}\NormalTok{(}
        \StringTok{"peripheral vascular disease"}\NormalTok{, }\StringTok{"peripheral gangrene"}\NormalTok{, }\StringTok{"-wrong answer"}\NormalTok{,}
        \StringTok{"intermittent claudication"}\NormalTok{, }\StringTok{"thromboangiitis obliterans"}\NormalTok{,}
        \StringTok{"thromboangiitis obliterans"}\NormalTok{, }\StringTok{"diabetic peripheral angiopathy"}\NormalTok{,}
        \KeywordTok{c}\NormalTok{(}\StringTok{"diabetes"}\NormalTok{, }\StringTok{"peripheral angiopathy"}\NormalTok{),  }\CommentTok{# single AND expression}
        \KeywordTok{c}\NormalTok{(}\StringTok{"buerger"}\NormalTok{,  }\StringTok{"disease presenile_gangrene"}\NormalTok{),}
            \StringTok{"-excepted"}\NormalTok{, }\CommentTok{# exclusion}
    \DataTypeTok{codes =} \KeywordTok{list}\NormalTok{(}\StringTok{"G73"}\NormalTok{),}
    \DataTypeTok{tests =} \OtherTok{NULL}\NormalTok{,}
    \DataTypeTok{drugs =} \KeywordTok{list}\NormalTok{(}\StringTok{"insulin"}\NormalTok{, }\StringTok{"diabet"}\NormalTok{, }\StringTok{"aspirin"}\NormalTok{)))}
\end{Highlighting}
\end{Shaded}

Code lists can be defined in a csv file with format as shown in table 2.
These files can then be imported to \texttt{MedicalDefinition} objects
using the \texttt{import\_definitions(input\_file = "path/to/file.csv")}
function.

\begin{longtable}[c]{@{}llll@{}}
\toprule
definition & status & items &\tabularnewline
\midrule
\endhead
terms & include & peripheral vascular disease &\tabularnewline
terms & include & peripheral gangrene &\tabularnewline
terms & exclude & wrong answer &\tabularnewline
terms & include & intermittent claudication &\tabularnewline
terms & include & thromboangiitis obliterans &\tabularnewline
terms & include & Diabetic peripheral angiopathy &\tabularnewline
terms & include & diabetes & peripheral angiopathy\tabularnewline
terms & include & buerger & disease presenile\_gengrene\tabularnewline
terms & exclude & excepted &\tabularnewline
codes & include & G73 &\tabularnewline
drugs & include & insulin &\tabularnewline
drugs & include & diabet &\tabularnewline
drugs & include & aspirin &\tabularnewline
\bottomrule
\end{longtable}

table 2: Example code list definition in csv format

The \texttt{MedicalDefinition} objects are then used to run searches
against lookup tables provided with EHRs via the
\texttt{build\_definition\_lists()} function:

\begin{Shaded}
\begin{Highlighting}[]
\NormalTok{## Use fileEncoding="latin1" to avoid any issues with non-ascii characters}
\NormalTok{medical_table <-}\StringTok{ }\KeywordTok{read.delim}\NormalTok{(}\StringTok{"Lookups/medical.txt"}\NormalTok{, }\DataTypeTok{fileEncoding =} \StringTok{"latin1"}\NormalTok{, }\DataTypeTok{stringsAsFactors =} \OtherTok{FALSE}\NormalTok{)}
\NormalTok{drug_table <-}\StringTok{ }\KeywordTok{read.delim}\NormalTok{(}\StringTok{"Lookups/product.txt"}\NormalTok{, }\DataTypeTok{fileEncoding =} \StringTok{"latin1"}\NormalTok{, }\DataTypeTok{stringsAsFactors =} \OtherTok{FALSE}\NormalTok{)}

\NormalTok{draft_lists <-}\StringTok{ }\KeywordTok{build_definition_lists}\NormalTok{(def, }\DataTypeTok{medical_table =} \NormalTok{medical_table, }\DataTypeTok{drug_table =} \NormalTok{drug_table)}
\end{Highlighting}
\end{Shaded}

\subsection{Unit conversion}\label{unit-conversion}

HbA1C tests for glycated haemoglobin are one of the best recorded
clinical tests in UK primary care databases, to a large extent because
of testing being incentivised under the UK Quality and Outcomes
Framework pay-for-performance scheme (Roland 2004; Kontopantelis et al.
2014). However, HbA1C data is not recorded in CPRD consistently.
Measurements may have been made in mmol/mol, mmol/L or mg/dL. Also the
closely analogous fructosamine test can also be converted into the same
units for direct comparison. The CPRD-specific
\texttt{cprd\_uniform\_hba1c\_values()} function accepts a single
argument of a dataframe in the CPRD ``Additional'' table form containing
only entity types for HbA1C and Fructosamine and converts any HbA1C and
fructosamine values to a common mmol/mol scale. Once this conversion has
taken place, the function also removes obvious mis-coding errors that
are far outside the possible range. A dataframe is returned with an
extra column \texttt{hba1c\_score}

\subsection{Exporting data to Stata
format}\label{exporting-data-to-stata-format}

Sometimes researchers may need to share data with others in the same
group who may not have \texttt{R} expertise. We have provided the
\texttt{to\_stata} function to export dataframes to stata dta format.
This function compresses a dataframe to reduce file size in the
following ways:

\begin{enumerate}
\def\labelenumi{\arabic{enumi}.}
\itemsep1pt\parskip0pt\parsep0pt
\item
  Date variables (as specified by the \texttt{date\_fields} argument)
  are converted to integer days from 1960-01-01 to avoid compatibility
  issues between \texttt{R} and Stata. An alternative origin can be set
  with the \texttt{origin} argument
\item
  Fields specified in the \texttt{integer\_fields} are converted from
  numeric to integer
\end{enumerate}

the \texttt{stata13} boolean argument indicates whether files should be
stored in Stata13 format (Using \texttt{readstata13::savedta13}) or in
Stata 12 compatible format (using \texttt{foreign::write.dta}). The
former includes a further compression step, similar to the
\texttt{compress} command in Stata.

\subsection{Working with temporary database
tables}\label{working-with-temporary-database-tables}

The size of EHR databases may require keeping intermediate data
extractions as database tables, rather than as in-memory \texttt{R}
dataframes. For example, extractions of clinical events for a common
condition such as diabetes or asthma will require the extraction of
millions of rows of data. These may be easily stored as temporary
database tables. This is also useful if you are working with a protected
database that you only have read-only access to. The \texttt{rEHR}
package has a suite of functions to deal with temporary database tables:

\begin{itemize}
\itemsep1pt\parskip0pt\parsep0pt
\item
  \texttt{temp\_table()} is used to construct temporary tables and is
  illustrated in section 3
\item
  \texttt{append\_to\_temp\_table()} appends rows to a temporary table
  based on a specified select statement
\item
  \texttt{to\_temp\_table()} exports a dataframe to a temporary database
  table
\item
  \texttt{drop\_temp\_table()} checks if a temporary table exists and
  then deletes if it does
\item
  \texttt{drop\_all\_temp\_tables()} drops all temporary tables from the
  database
\end{itemize}

Note that temporary tables are only associated with the currently open
database connection. This means that functions capable of parallel
processing (e.g. \texttt{select\_by\_year()}) can only be used in the
single core mode (i.e.~set \texttt{cores = 1}) since multicore processes
open up multiple parallel connections.

\section{6. Setting EHR type}\label{setting-ehr-type}

In the final section we discuss the \texttt{.ehr} environment used to
define the EHR database being used and how this can be set to work with
different databases.

In many of the functions in this package, specific tables and variables
in the database need to be accessed. A particular database system, such
as CPRD, will have its own schema describing the organisation of the
data within it. To simplify the functions in this package, we have opted
to include an interface to the database schema in the form of an
environment, \texttt{.ehr}, that is accessed by the various analysis
functions in order to extract the correct data from the correct place in
the database. This is effectively a list of attributes relating to the
EHR system being used. For example there is an attribute specifying the
patient id variable in the database. By default, a schema environment
for CPRD is loaded when the package is loaded via a call to
\texttt{set\_CPRD()}. we have provided accessor functions to get and set
attributes in the \texttt{.ehr} environment. It is preferable to use
these accessor functions rather than setting elements directly. A list
of all of the attributes is provided by the
\texttt{list\_EHR\_attributes()} function. For example:

\begin{Shaded}
\begin{Highlighting}[]
\KeywordTok{list_EHR_attributes}\NormalTok{()}
\end{Highlighting}
\end{Shaded}

\begin{verbatim}
##  [1] "birth_year"      "cohort"          "date_fields"    
##  [4] "ehr_medcode"     "EHR_name"        "event_date"     
##  [7] "lookup"          "patient_id"      "practice_id"    
## [10] "raw_date_format" "tables"          "year_origin"
\end{verbatim}

The values of individual attributes can be accessed with the
\texttt{get\_EHR\_attribute()} function:

\begin{Shaded}
\begin{Highlighting}[]
\KeywordTok{get_EHR_attribute}\NormalTok{(patient_id) }\CommentTok{# gives the attribute for patient ids}
\end{Highlighting}
\end{Shaded}

\begin{verbatim}
## [1] "patid"
\end{verbatim}

\begin{Shaded}
\begin{Highlighting}[]
\KeywordTok{get_EHR_attribute}\NormalTok{(date_fields) }\CommentTok{# fields in the database stored as dates}
\end{Highlighting}
\end{Shaded}

\begin{verbatim}
##        event        entry    last_coll    up_to_std    first_reg 
##  "eventdate"    "sysdate"        "lcd"        "uts"        "frd" 
##  current_reg transfer_out        death 
##        "crd"        "tod"  "deathdate"
\end{verbatim}

\begin{Shaded}
\begin{Highlighting}[]
\KeywordTok{get_EHR_attribute}\NormalTok{(cohort) }\CommentTok{# variables used in cohort construction}
\end{Highlighting}
\end{Shaded}

\begin{verbatim}
## $start_criteria
## [1] "crd" "uts"
## 
## $end_criteria
## [1] "tod"       "deathdate" "lcd"
\end{verbatim}

Individual attribute values can be set using the
\texttt{set\_EHR\_Attribute()} function:

\begin{Shaded}
\begin{Highlighting}[]
\KeywordTok{set_EHR_attribute}\NormalTok{(patient_id, }\DataTypeTok{value =} \StringTok{"PATIENT"}\NormalTok{) }\CommentTok{# set the patient id attribute}
\KeywordTok{get_EHR_attribute}\NormalTok{(patient_id)}
\end{Highlighting}
\end{Shaded}

\begin{verbatim}
## [1] "PATIENT"
\end{verbatim}

The default settings can be reverted to using the \texttt{set\_CPRD()}
function:

\begin{Shaded}
\begin{Highlighting}[]
\KeywordTok{set_CPRD}\NormalTok{()}
\end{Highlighting}
\end{Shaded}

\begin{verbatim}
## Using CPRD settings
\end{verbatim}

\begin{Shaded}
\begin{Highlighting}[]
\KeywordTok{get_EHR_attribute}\NormalTok{(patient_id)}
\end{Highlighting}
\end{Shaded}

\begin{verbatim}
## [1] "patid"
\end{verbatim}

The \texttt{.ehr} environments will allow for the simple definition of
interfaces to other EHR systems, via the construction of new setting
functions.

\section{7. Conclusion}\label{conclusion}

Working with structured EHR data requires a combination of computational
and statistical expertise. The \texttt{rEHR} package greatly simplifies
and accelerates the extraction and processing of coded data from EHR
databases, enabling researchers to spend more time on their analyses,
time that would otherwise be consumed with laborious preparation of
research-ready data. The workflow is straightforward, amounting to a
flat series of function calls rather than a complex set of nested loops,
therefore errors are much more easily spotted and fixed. The combination
of SQL native databases, optimised data manipulation packages and
multicore functionality results in a package that runs many times faster
than equivalent code.

\subsection{Limitations and future
work}\label{limitations-and-future-work}

Although rEHR is currently only tested with CPRD data, the \texttt{.ehr}
environment system will allow it to be easily linked to other EHR
databases. Future versions of the \texttt{rEHR} software will include:

\begin{itemize}
\itemsep1pt\parskip0pt\parsep0pt
\item
  Implementation of the \texttt{repsample} algorithm for representative
  sampling of practices (Kontopantelis 2013).
\item
  Iterative proportional fitting for matching on population
  characteristics between different EHR databases (Springate et al.
  (2015) Appendix 2).
\item
  A robust algorithm for determining smoking status.
\item
  Interfaces to other EHR systems, in particular UK primary care
  databases such as THIN, QResearch and Research One.
\item
  Uniform units functions for other clinical measurements such as blood
  pressure, cholesterol and serum creatinine.
\end{itemize}

\section{Acknowledgements}\label{acknowledgements}

This work was funded by the National Institute for Health Research
(NIHR) School for Primary Care as part of the project ``An analytical
framework for increasing the efficiency and validity of research using
primary care databases'' and by MRC Health eResearch Centre Grant
MR/K006665/1.

\section*{References}\label{references}
\addcontentsline{toc}{section}{References}

Ainsworth, John, James Cunningham, and Iain Buchan. 2012. ``ELab:
Bringing Together People, Data and Methods to Enhance Knowledge
Discovery in Healthcare Settings.'' \emph{Studies in Health Technology
and Informatics} 175: 39--48.

Bechhofer, Sean, Iain Buchan, David De Roure, Paolo Missier, John
Ainsworth, Jiten Bhagat, Philip Couch, et al. 2013. ``Why Linked Data Is
Not Enough for Scientists.'' \emph{Future Generation Computer Systems}
29 (2). Elsevier: 599--611.

Bendix Carstensen, Esa Laara, Martyn Plummer, and Michael Hills. 2014.
\emph{Epi: A Package for Statistical Analysis in Epidemiology}.
\url{http://CRAN.R-project.org/package=Epi}.

``Clinical Practice Research Datalink (CPRD) GOLD.''
\url{http://www.cprd.com/dataAccess/default.asp\#OnlineDataGOLD}.

Danaei, Goodarz, Luis A García Rodríguez, Oscar Fernández Cantero, Roger
Logan, and Miguel A Hernán. 2013. ``Observational Data for Comparative
Effectiveness Research: An Emulation of Randomised Trials of Statins and
Primary Prevention of Coronary Heart Disease.'' \emph{Statistical
Methods in Medical Research} 22 (1): 70--96.
doi:\href{http://dx.doi.org/10.1177/0962280211403603}{10.1177/0962280211403603}.

Davé, Shreya, and Irene Petersen. 2009. ``Creating Medical and Drug Code
Lists to Identify Cases in Primary Care Databases.''
\emph{Pharmacoepidemiology and Drug Safety} 18 (8). John Wiley \& Sons,
Ltd.: 704--7.
doi:\href{http://dx.doi.org/10.1002/pds.1770}{10.1002/pds.1770}.

Gelfand, Joel M, Andrea L Neimann, Daniel B Shin, Xingmei Wang, David J
Margolis, and Andrea B Troxel. 2006. ``Risk of Myocardial Infarction in
Patients with Psoriasis.'' \emph{Jama} 296 (14). American Medical
Association: 1735--41.

Grothendieck, G. 2014. \emph{Sqldf: Perform SQL Selects on R Data
Frames}. \url{http://CRAN.R-project.org/package=sqldf}.

James, David A., Seth Falcon, and the authors of SQLite. 2013.
\emph{RSQLite: SQLite Interface for R}.
\url{http://CRAN.R-project.org/package=RSQLite}.

Kontopantelis, Evangelos. 2013. ``A Greedy Algorithm for Representative
Sampling: Repsample in Stata.'' \emph{Journal of Statistical Software}
55 (1).

Kontopantelis, Evangelos, David A Springate, David Reeves, Darren M
Ashcroft, Martin Rutter, Iain Buchan, and Tim Doran. 2014. ``Glucose,
Blood Pressure and Cholesterol Levels and Their Relationships to
Clinical Outcomes in Type 2 Diabetes: A Retrospective Cohort Study.''
\emph{Diabetologia}. Springer, 1--14.

Olier, Ivan, David A Springate, David Reeves, Darren M Ashcroft, Tim
Doran, Sioban Reilly, and Evangelos Kontopantelis. 2015. ``Modelling
Conditions in a Primary Care Database: An Application to Severe Mental
Illness with the Clinical Practice Research Datalink.'' \emph{Submitted
BMJ Open}.

Overhage, J Marc, and Lauren M Overhage. 2013. ``Sensible Use of
Observational Clinical Data.'' \emph{Statistical Methods in Medical
Research} 22 (1): 7--13.
doi:\href{http://dx.doi.org/10.1177/0962280211403598}{10.1177/0962280211403598}.

Parisi, Rosa, Martin K Rutter, Mark Lunt, Helen S Young, Deborah PM
Symmons, Christopher EM Griffiths, and Darren M Ashcroft. 2015.
``Psoriasis and the Risk of Major Cardiovascular Events: Cohort Study
Using the Clinical Practice Research Datalink.'' \emph{Journal of
Investigative Dermatology}. Nature Publishing Group.

Perlis, R. H., D. V. Iosifescu, V. M. Castro, S. N. Murphy, V. S.
Gainer, J. Minnier, T. Cai, et al. 2012. ``Using Electronic Medical
Records to Enable Large-Scale Studies in Psychiatry: Treatment Resistant
Depression as a Model.'' \emph{Psychological Medicine} 42 (01): 41--50.
doi:\href{http://dx.doi.org/10.1017/S0033291711000997}{10.1017/S0033291711000997}.

R Core Team. 2014. \emph{R: A Language and Environment for Statistical
Computing}. Vienna, Austria: R Foundation for Statistical Computing.
\url{http://www.R-project.org/}.

Reeves, David, David A Springate, Darren M Ashcroft, Ronan Ryan, Tim
Doran, Richard Morris, Ivan Olier, and Evangelos Kontopantelis. 2014.
``Can Analyses of Electronic Patient Records Be Independently and
Externally Validated? The Effect of Statins on the Mortality of Patients
with Ischaemic Heart Disease: A Cohort Study with Nested Case--control
Analysis.'' \emph{BMJ Open} 4 (4).
doi:\href{http://dx.doi.org/10.1136/bmjopen-2014-004952}{10.1136/bmjopen-2014-004952}.

Richardson, D B. 2004. ``An Incidence Density Sampling Program for
Nested Case-Control Analyses.'' \emph{Occupational and Environmental
Medicine} 61 (12): e59.
doi:\href{http://dx.doi.org/10.1136/oem.2004.014472}{10.1136/oem.2004.014472}.

Roland, Martin. 2004. ``Linking Physicians' Pay to the Quality of
Care---a Major Experiment in the United Kingdom.'' \emph{New England
Journal of Medicine} 351: 1448--54.

Springate, David A, Darren M Ashcroft, Evangelos Kontopantelis, Tim
Doran, Ronan Ryan, and David Reeves. 2015. ``Can Analyses of Electronic
Patient Records Be Independently and Externally Validated? Study 2---the
Effect of \(B\)-Adrenoceptor Blocker Therapy on Cancer Survival: A
Retrospective Cohort Study.'' \emph{BMJ Open} 5 (4).
doi:\href{http://dx.doi.org/10.1136/bmjopen-2014-007299}{10.1136/bmjopen-2014-007299}.

Springate, David A., Evangelos Kontopantelis, Darren M. Ashcroft, Ivan
Olier, Rosa Parisi, Edmore Chamapiwa, and David Reeves. 2014.
``ClinicalCodes: An Online Clinical Codes Repository to Improve the
Validity and Reproducibility of Research Using Electronic Medical
Records.'' \emph{PLoS ONE} 9 (6). Public Library of Science: e99825.
doi:\href{http://dx.doi.org/10.1371/journal.pone.0099825}{10.1371/journal.pone.0099825}.

Wickham, Hadley, and Winston Chang. 2014. \emph{Devtools: Tools to Make
Developing R Code Easier}.
\url{http://CRAN.R-project.org/package=devtools}.

Wickham, Hadley, and Romain Francois. 2015. \emph{Dplyr: A Grammar of
Data Manipulation}. \url{http://CRAN.R-project.org/package=dplyr}.

\end{document}
